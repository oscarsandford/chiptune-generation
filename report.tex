\documentclass{article}
\usepackage[T1]{fontenc} % add special characters (e.g., umlaute)
\usepackage[utf8]{inputenc} % set utf-8 as default input encoding
\usepackage{ismir,amsmath,cite,url}
\usepackage{graphicx}
\usepackage{color}

% UNCOMMENT THE TWO LINES BELOW TO SEE LINE NUMBERS. I think it is better without the line numbers.
% \usepackage{lineno}
% \linenumbers

\title{Track Extension and Generation using Markov chains}

% Three addresses
% --------------
\threeauthors
  {Oscar Sandford} {University of Victoria \\ {\tt oscarsandford@uvic.ca}}
  {Colson Demedeiros} {University of Victoria \\ {\tt cdemedeiros@uvic.ca}}
  {Jae Park} {University of Victoria \\ {\tt jaehyunpark@uvic.ca}}

\sloppy
\begin{document}
\maketitle

\begin{abstract}
Write this last.
\end{abstract}

\section{Introduction}
Music is typically constructed by humans, for humans. However, machines are becoming more adept at revealing patterns in the way musicians craft their chords. 
This enables humans to create their own music, and then let the machine take over the task of composer. Our project aims to understand and implement a framework 
for simple 8-bit music generation through temporal inference techniques. In this paper, we explore previous implementations and existing literature regarding music 
generation, and present techniques for designing generative Markov models to extend music as well as create derivative musical works.

\section{Related Work}
\subsection{Markov Models}
Yanchenko and Mukherjee explored the use of Hidden Markov Models (HMMs) to compose classical music, finding proficiency in generating consonant harmonies, but 
lacking melodic progression \cite{yanchenko_2017}. Indeed, the models were found to learn the harmonic hidden states quite well, in some cases leading to overfitting. 
HMMs have also found use in chorale harmonization, where a given observable melody uses inference to derive hidden harmonies to complement it \cite{allan_2005}. 

Walter and Van Der Merwe's methods involve representing the chord duration, chord progression, and rhythm progression with first or higher-order Markov chains, whereas 
the overlaying melodic arc is represented by a HMM \cite{walter_2011}. This separation works well to reduce the processing power needed for music generation, but the 
independent learning of each component leads to less cohesive compositions. Generating music is generally done by sampling a statistical model \cite{conklin_2003}.
However, we want to create music that does not only simply replicate the training data, but also creates cohesive pieces in a more natural way. 

Shapiro and Huber's approach to music generation simply uses Markov chains, no hidden states \cite{shapiro_huber_2021}. In their work, the states represent sound 
objects with attributes such as pitch, octave, and duration. Their results show that human-composed pieces can be closely replicated using the simpler Markov chains. 
Further, they have attached their implementation in their paper. We will consider this work when constructing our own implementation.

Corrêa and Jungling suggest using Markov chains of different orders to predict classical music. By using stochastic models to analyze different classcal songs and styles,
the computer can even capture subtle and intuitive features such as style of the composer. Although this paper focuses on classcial music and its prediction, the authors 
stress that its application to other music genres should be straightforward \cite{correa_jungling_small_2020}.
The only requirement is a MIDI file with a good quality, as it should be used for markov chains can properly estimate the music's pattern.

\subsection{Data Format}
The papers previously mentioned use the MIDI file format to write digital music. This format appears to be the standard for digital music creation \cite{midi_format}. 
One of MIDI's drawbacks is that it cannot store vocals \cite{cataltepe_2007}. This is of no concern to us, as we will only be attempting to generate instrumental 
compositions. Additionally, successful approaches to melody extraction from MIDI files \cite{ozcan_2005} make assurances that this will be an adequate medium for the 
music our models will generate. MusicXML is a standard file format for storing sheet music, just as MP3 is for recordings \cite{musicxml_2022}. As both are commonly used 
standards, we plan to use MusicXML for input data and write our output to MIDI files.

\section{Timeline}
We expect to have 3 milestones for this project:
\begin{enumerate}
  \item \textbf{Milestone 1} (February 18th): Convert music to MusicXML files using AnthemScore. Devise sound object structure for Markov states or other architecture. 
  Write a parser to transform the MusicXML data to sound objects. 
  \item \textbf{Milestone 2} (March 21st): Apply Markov chain algorithm to the music samples in order to train the model. We plan to refer to various works that we found.
  By this point, our models should be able to replicate the music tracks given as input. By this point our program will be able to extend music.
  \item \textbf{Milestone 3} (April 7th): Tweak the models to generate more original variations. Enhance the quality of the music. Possibly a GUI for aesthetics as well, 
  if time permits.
\end{enumerate}

\section{Task Delegation}
In order to figure out the best approach and gather a plethora of sources, we are each looking at various sources related to music data parsing and music generation, 
from theoretical papers to Python libraries. Colson and Jae are finding classic 8-bit tracks that we will use to train our models on. Oscar has set up a GitHub repository 
to include written work as well as source code, and made outlines for the final report. Moving forward, Colson will be responsible for finding and converting music to 
MusicXML format, and designing a parser to turn notes into data. Jae and Oscar will work on designing the sound object states and programming the generative Markov models. 

\section{Resources}
\subsection{Tools}
Stacher\footnote{https://stacher.io} is a frontend GUI for YT-DLP, which is a command line downloader that can be installed for converting Youtube videos to MP3 files.
While YT-DLP works just fine, Stacher makes it very easy to convert YouTube links to various file formats(MP3, WAV, AAC, etc.) and save them on your current device. This is 
the software we will use to get the songs as MP3 files saved and put into AnthemScore to be converted into XML files.

AnthemScore\footnote{https://www.lunaverus.com/} is a music transcription software that converts WAV, MP3, and other audio formats into sheet music using an advanced neural 
network. The sheet music can then be exported to various other formats such as PDF, MIDI, or XML. The main use of this software is to easily obtain XML files that we will 
use for data in the Markov chains. While AnthemScore is a great tool to aquire relatively true XML files of songs, the artificial intelligence that powers it does occasionally 
miss notes. What this means for us is that without changing the notes that are detected and placed by the AnthemScore software, it may miss notes or sounds that exist within 
the original song. Despite this, as long as it is mostly accurate and the majority of notes are in place, it shouldn't affect the overall process of the project. We will avoid 
songs that don't transform accurately into MusicXML. Preliminary coding will be done in Jupyter Notebooks using Python. 

\subsection{Data Sets}
We are planning to use mostly music from various classic video games, including Super Mario Brothers \cite{kondo_2009},
Undertale \cite{fox_2017} and Legend of Zelda \cite{nakatsuka_2009}.

% END

% Add bibtext citations to the file `report_bibliography.bib`.
\bibliography{report_bibliography}
\end{document}
